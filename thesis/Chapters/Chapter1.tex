% Chapter 1

\chapter{Introduction} % Main chapter title

\label{Chapter1} % For referencing the chapter elsewhere, use \ref{Chapter1} 

\lhead{Chapter 1. \emph{Introduction}} % This is for the header on each page - perhaps a shortened title

%----------------------------------------------------------------------------------------



The purpose of this master's project is to quantify large-scale functional and structural brain networks and the comparison to resting-state functional Magnetic Resonance Imaging (fMRI). The functional brain networks are derived from simulated blood oxygen level dependent (BOLD) signals, whereas the structural brain networks are obtained from diffusion weighted magnetic resonance imaging (DW-MRI). The project uses experimental results combined with modeling approaches and implements methods drawn from nonlinear and network science. 

Large-scale functional brain connectivity maps are networks of brain regions based on functional interactions, i.e. any co-activation between these regions \citep{BIS95, BRE10b, DAM06}. In a typical fMRI experiment,  functional connections are obtained from pre-defined brain regions, whose corresponding time-series of BOLD activity display significant correlations at low-frequencies ($< 0.1$ Hz). Measured BOLD activity patterns are complex, but they are also highly structured and robust. Moreover, structural motives of the correlated activity have been reported not only during brain's activation paradigm, but also in the course of so-called resting state, i.e. under no stimulation and in the absence of any stimulus-driven task. Despite its importance, the underlying biophysical process of the resting activity of the brain has not yet been fully resolved. One of the main objectives in this project is to capture resting state BOLD fluctuations by modeling time-series and BOLD activity for nerve cell populations in the brain. 

DW-MRI technique estimates the structural connection probabilities among brain regions indirectly by investigating the diffusion direction of water molecules. The direction of the fiber tracks in white matter depends indirectly on the diffusion of water molecules. A DW-MRI experiment approximates the existence of a fiber track between regions of interest. Both anatomical and functional brain connectivity maps used in this project are empirically obtained from the same cortical and sub-cortical regions. For this purpose, the brain images are partitioned into $N=90$ regions based on the Tzourio-Mazoyer brain atlas using the automated anatomical labeling (AAL) method \citep{TZO02}. In fact, it will be discussed how functional connections between brain regions could arise from complex structural connections. 

Despite important progress over the past few years, the way how functional connectivity arises from the complex anatomical connectivity still remains poorly understood \citep{VUK14}. Existing models of resting-brain dynamics hypothesize that functional interactions result from a complex interplay between intrinsic brain dynamics and underlying structural connections \citep{RUB09}. In particular, these models explore the range of conditions at which functional networks emerge from anatomical connections, the role of multiple time-scales in the formation of functional connectivity networks \citep{HON07}, time delays in the signal propagation between the network nodes as well as the system noise \citep{GHO08, GHO08a}, local network oscillations \citep{DEC09, CAB11} and structural disconnection \citep{CAB12}.

The neuronal activity model of the resting activity of brain is built on FitzHugh-Nagumo (FHN) oscillators as  previously proposed in \citep{VUK13, VUK14}. The temporal dynamics of membranous potential of each AAL region (a single \textit{node}) is simulated with FHN model. The model parameters are tuned in such a way that the dynamics of nerve cells in each node exhibit type-II excitability. One of the key ingredients of the network model is the \textit{coupling strength} $c$, which scales mutual time-delayed functional interactions among brain regions globally. The time-delay in the model appears as a natural consequence of a finite speed of signal propagation along axons. Therefore, the \textit{signal velocity} $v$, representing biophysically realistic axonal signal propagation \citep{GHO08, GHO08a, DEC09} is considered to be another significant ingredient for FHN dynamics. 

FHN model is used to extract time-series for AAL regions based on functional and anatomical brain networks. However, high frequency FHN oscillations need to be tuned into slow fluctuations in order to capture BOLD signals observed in fMRI. The BOLD activity is inferred via the Baloon-Windkessel hemodynamic model, which takes the simulated neuronal activity time-series as an input and gives the simulated BOLD activity as an output \citep{FRI00}. In this project, the hemodynamic model is applied on FHN time-series extracted from functional and anatomical brain networks. Then, any correlation matrix based on simulated BOLD signal is statistically compared to the empirical correlation matrix based on fMRI-BOLD.      

Functional and anatomical brain networks are constructed after binarizing via thresholding the fMRI-BOLD and DW-MRI data, respectively. The spatial dynamics of each network is identified in dependence on the threshold level; $r$ for functional connectivity (FC) and $p$ for anatomical connectivity (AM). Statistical characterization of brain networks, using methods from graph theory, has revealed some of their key topological properties such as small worldness, modularity or resilience to attacks, that is node or link removal \citep{BUL09}. This project studies these properties both for functional and structural brain networks, which arise from modeled intrinsic brain dynamics. In particular, I aim to explore such conditions that distinguish obtained network topologies from that of random networks. Several randomization procedures are considered. They include, but are not limited to random networks of Erd\H{o}s-R\'{e}nyi-type with the same number of nodes and links as in the empirically derived case. This approach is expected to provide a deeper insight into the underlying processes involved in the observed functional connectivity and their relations to the coupling topology, i.e., brain structural connectivity.


The rest of the master's thesis is organized in the following order : Empirical data sets of FC and AC matrices are introduced in Section 2.1, and it is further extended with brain graph construction based on these connectivity maps in Section 2.2. Section 2.3 explains the randomization methods used to build random graphs. Characterization of all constructed networks by using methods of network science \citep{RUB09, STA10, NEW10, RUB11} is done by quantifying global and local network properties such as network density, clustering coefficient and small-worldness in Section 2.4. Temporal dynamics emerging from the FitzHugh-Nagumo model and the Balloon-Windkessel hemodynamic model \citep{FRI00} are described in Sections 2.5 and 2.6. Section 3.1 illustrates simulated neuronal activity results, i.e. the statistical comparison of FHN simulated brain graphs to the fMRI-BOLD data in parameter space of $(r,c)$ and $(r,v)$.  Section 3.2 demonstrates simulated BOLD fluctuations applied on functional and anatomical brain graphs. Section 3.3 compares the modeled temporal dynamics of brain graphs to that of random networks with a statistical quantification. Finally, Section 4 concludes the master thesis with the research proposals: it is possible to explore such regions on parameter space, where \textit{i)} BOLD fluctuations are captured through structural brain connections \textit{ii)} network measures and temporal dynamics of brain graphs are different from that of random graphs. 
 
 

