% Chapter 1

\chapter{Introduction} % Main chapter title

\label{Chapter1} % For referencing the chapter elsewhere, use \ref{Chapter1} 

\lhead{Chapter 1. \emph{Introduction}} % This is for the header on each page - perhaps a shortened title

%----------------------------------------------------------------------------------------



The purpose of this master's project is to quantify large-scale functional and structural brain networks and the comparison to resting-state functional Magnetic Resonance Imaging (fMRI). The functional brain networks are derived from simulated Blood-Oxygen-Level-Dependent (BOLD) signals, whereas the structural brain networks are obtained from Diffusion-Weighted Magnetic Resonance Imaging (DW-MRI). The project uses experimental results combined with modelling approaches and implements methods drawn from nonlinear and network science. 

The functional connectivities in a typical fMRI experiment are obtained from pre-defined brain regions, whose corresponding time-series of BOLD activity display significant correlations at low-frequencies ($<$ 0.1 Hz). The measured activity patterns are correlated and complex, but highly structured and robust. They have been observed also for brains in resting state, i.e. under no stimulations and in the absence of any stimulus-driven task. The fMRI-BOLD empirical data used in this project is obtained from the \textit{1000 Functional Connectome Project} website \url{http://www.nitric.org/}).  
This data set is also referred as functional connectivity matrix (FCM) revealing the correlated activities among the 90 cortical and sub-cortical regions with the automated anatomical labelling (AAL) template as given in Table 1 in Appendix.

The DW-MRI technique estimates the anatomical connection probabilities among brain regions indirectly by investigating the diffusion direction of water molecules. The direction of the fiber tracks in white matter depends indirectly on the diffusion of water molecules. A DW-MRI experiment approximates the existence of a fiber track between regions of interest. The anatomical connection probability (ACP) matrix for the 90 anatomically labelled brain regions considered in this project is obtained from the study of Iturria-Medina et. al. (2008) [1]. Both anatomical and functional connectivities are considered for the same cortical and sub-cortical regions. 

[1] Iturria-Medina et. al. (2008)

Statistical characterization of the functional brain networks, using methods from graph theory, has revealed some of their key topological properties such as small worldness, modularity or resilience to the attacks [BUL09]. This project studies these properties both for functional and structural brain networks arising from modelled intrinsic brain dynamics. In particular, such conditions that distinguish obtained network topologies from that of random networks will be explored. Several randomization procedures will be considered. They include, but are not limited to random networks of Erdos-Renyi-type with the same number of nodes and links as in the empirically derived case. This approach will provide a deeper insight into the underlying processes involved in the observed functional connectivity and their relations to the coupling topology, i.e., brain structural connectivity.

 
Despite important progress over the past few years, the way how functional connectivity arises from the complex anatomical connectivity still remains poorly understood [VUK14]. Existing models of resting-brain dynamics hypotheise that functional interactions result from a compley interplay between intrinsic brain dynamics and underlying structural connections [RUB09]. In particular, these models explore the range of conditions at which functional networks emerge from anatomical connections, the role of multiple time-scales in the formation of functional-commectivity networks [HON07], time delays in the signal propagation between the network nodes as well as the system noise [GHO08, GHO08a], local network oscillations [DEC09, CAB11] and structural disconnection [CAB12].

In this project, the model of resting-state brain activity will be based on the models previously proposed in [VUK13, VUK14]. The key ingredients are coupling topologies of the time-delayed functional interactions which are scaled by the global coupling strength $c$ and  are subject to the Gaussian white noise. The time-delay in the model arises as the natural consequence of a finite speed of signal propagation along axons. Therefore, velocity $v$, representing biophysically realistic axonal signal propagation [GH008, GHO08a, DEC09] is another important ingredient of the model. 

The rest of the master's thesis is organized in the following order : Section 2 will introduce the empirical data sets of FCM and ACM used in this project. This section will further explain the randomization methods used to construct random graphs. Characterization of those brain networks by using methods of network science [RUB09, STA10, NEW10, RUB11] will be done by quantifying global and local network properties such as network density, clustering coefficient, small-worldness etc.. Section 2 will introduce the FitzHugh-Nagumo (FHN) model of neuronal interactions [14,15 VUK13ten bak!] and Balloon-Windkessel hemodynamic model [16 -VUK13ten bak]. Section 3 will illustrate the effect of coupling strength $c$ and signal propagation velocity $v$ on the empirically derived brain networks and reshuffled networks. Additionally, such parameter regions will be identified at which the network properties of the emprirical data differ from that of random graphs. Section 4 will conclude the master's thesis. 

 
 

